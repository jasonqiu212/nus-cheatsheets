\documentclass{article}
\usepackage[a4paper, margin=3mm, landscape]{geometry}
\usepackage{multicol}
\usepackage{xcolor}
\usepackage{enumitem}
\usepackage{amsmath}

\pdfinfo{
  /Title (ST1131.pdf)
  /Creator (TeX)
  /Producer (pdfTeX 1.40.0)
  /Author (Jason Qiu)
  /Subject (ST1131)
  /Keywords (ST1131, nus, cheatsheet,pdf)
}

\pagestyle{empty}
\setcounter{secnumdepth}{0}
\setlength{\columnseprule}{0.25pt}

% Redefine section commands to use less space
\makeatletter
\renewcommand{\section}{\@startsection{section}{1}{0mm}%
  {-1ex plus -.5ex minus -.2ex}%
  {0.5ex plus .2ex}%x
{\normalfont\large\bfseries}}
\renewcommand{\subsection}{\@startsection{subsection}{2}{0mm}%
  {-1explus -.5ex minus -.2ex}%
  {0.5ex plus .2ex}%
{\normalfont\normalsize\bfseries}}
\renewcommand{\subsubsection}{\@startsection{subsubsection}{3}{0mm}%
  {-1ex plus -.5ex minus -.2ex}%
  {1ex plus .2ex}%
{\normalfont\small\bfseries}}%
\makeatother

% Adjust spacing for all itemize/enumerate
\setlength{\leftmargini}{0.5cm}
\setlength{\leftmarginii}{0.5cm}
\setlist[itemize,1]{leftmargin=2mm,labelindent=1mm,labelsep=1mm}
\setlist[itemize,2]{leftmargin=4mm,labelindent=1mm,labelsep=1mm}

% Font
\renewcommand{\familydefault}{\sfdefault}

% Define colors for math formulas
\definecolor{myblue}{cmyk}{1,.72,0,.38}
\everymath\expandafter{\the\everymath \color{myblue}}

% -----------------------------------------------------------------------
\begin{document}
\begin{multicols}{4}
\footnotesize

% Title box
\begin{center}
  \fbox{
    \parbox{0.8\linewidth}{
      \centering \textcolor{black}{
        {\Large\textbf{ST1131}} \\
        \normalsize{AY21/22 Sem 2}} \\
        {\footnotesize \textcolor{gray}{github.com/jasonqiu212}}
    }
  }
\end{center}

\section{01. Exploratory Data Analysis}

\subsection{Types of Variables}
\begin{itemize}
  \item \textbf{Quantitative}: Discrete vs. Continuous
  \item \textbf{Categorical}: Ordinal vs. Nominal
  \item How to tell difference: Is distance between 2 points meaningful?
\end{itemize}

\subsection{1 Variable}

\subsubsection{Frequency Table - Categorical}
\begin{itemize}
  \item \textbf{Proportion} - aka relative frequency. $\frac{\text{\# of obs. in 1 cat.}}
  {Total \# of obs.}$
  \item \textbf{Modal Frequency} - Category with highest frequency
  \item Summarizing: Modal category and its proportion 
\end{itemize}

\subsubsection{Bar Plots - Categorical, Visual}
\begin{itemize}
  \item Summarizing: Modal category and its proportion, Categories with high/low
  proportions, Mention trends if ordinal
\end{itemize}

\subsubsection{Histogram - Quantitative}
\begin{itemize}
  \item Summarizing: Gaps/Outliers, Unimodal/Bimodal/Multimodal, Symmetric/Skewed
  \item Skewed left: Left tail is longer. Skewed right: Right tail is longer.
\end{itemize}

\subsubsection{Describing Center}
\begin{itemize}
  \item \textbf{Mean} - $\bar{X} = \frac{1}{n} \sum_{i=1}^{n} x_{i}$
  \begin{itemize}
    \item Linear Transformation: $\hat{Y} = b \hat{X} + a$
    \item Sensitive to outliers, unlike median
  \end{itemize}
  \item \textbf{Median} - $X_{(0.5)}$
  \item If $\bar{X} > X_{(0.5)}$, skewed right. If $\bar{X} < X_{(0.5)}$, skewed left
\end{itemize}

\subsubsection{Describing Variability}
\begin{itemize}
  \item \textbf{Range} - Sensitive to outliers
  \item \textbf{Variance} - $S^{2} = \frac{1}{n - 1} \sum_{i=1}^{n} (x_{i} - \bar{x})^2$
  \item \textbf{Standard deviation} - $sd = \sqrt{S^2}$
  \begin{itemize}
    \item Linear Transformation: $S_{y}^2 = b^2s_{x}^2$ $S_{y} = |b|s_{x}$
  \end{itemize}
  \item \textbf{Inter-quartile Range (IQR)} - $Q_{3} - Q_{1}$
  \begin{itemize}
    \item \textbf{Quantile} - ($q_{p}$) Value such that p of observations are below
    \item Lower quartile ($Q_{1}$), Median ($Q_{2}$), Upper quartile ($Q_{3}$)
  \end{itemize}
  \item If symmetric, mean and variance. If skewed, median and IQR.
\end{itemize}

\subsubsection{Boxplot - Variability}
  \begin{itemize}
    \item Includes: Min, Q1, Q2, Q3, Max
    \item \textbf{Outliers} - $< Q_{1} - 1.5IQR \text{ or } > Q_{3} + 1.5IQR$
    \item \textbf{Max/min Whisker Reach} - Boundary of outliers
    \item \textbf{Upper/lower Whisker} - Min/max obs. excluding outliers
    \item Does not show features of distribution. If unimodal, can show skewness.
    \item Summarizing: Median, Outliers, Compare medians and IQR if $>1$ boxplots
  \end{itemize}

\subsection{2 Variables (Response Variable vs. Explanatory Variable)}

\subsubsection{2 Categorical Variables}
\textbf{Bar Plots} \\
\textbf{Contingency Table}
\begin{itemize}
  \item \textbf{Conditional Percentage} - \% out of total
  \item \textbf{Join Percentage} - \% out of some group. Use explanatory as group.
  \item Be careful of phrasing (Eg. Ppl w/o cancer of PMH users vs. PMH users of 
  those w/o cancer)
  \item \textbf{Relative Risk} - Ratio of 2 percentages. (Eg. \% of cancer in PMH users
  is 1.24 times the \% of cancer in non-PMH users)
\end{itemize}

\subsubsection{1 Categorical and 1 Quantitative}
\textbf{2 Boxplots} - Split by categories

\subsubsection{2 Quantitative Variables}
\textbf{Scatter Plot}
\begin{itemize}
  \item Summarizing: Pos./neg. association, Linear, Constant variability, Outliers
\end{itemize}
\textbf{Correlation} - $r \in [-1, 1]$
\begin{itemize}
  \item 2 variables have same correalation, no matter $x \sim y$ or $y \sim x$
  \item Correlation is linear, when $r = \pm 1$
\end{itemize}

\section{02. Data Collection}

\begin{itemize}
  \item \textbf{Confounding Variable} - Related to exp. and resp. variable. 
  Confounds their association. Observed.
  \item \textbf{Lurking Variable} - Unobserved
  \item \textbf{Experimental Study} - Assign subjects to treatments and observe 
  response variable
  \begin{itemize}
    \item Pros: Control over lurking variables 
    \item Cons: Costly, Unethical
  \end{itemize}
  \item \textbf{Observational Study} - Explanatory and response variable observed 
  for subjects. No treatments. 
\end{itemize}

\subsubsection{Sample Survey}
\begin{enumerate}
  \item Identify population
  \item Compile \textbf{sampling frame} - Where sample is from
  \item \textbf{Sampling design} - How to choose subjects from sampling frame
  \begin{itemize}
    \item \textbf{Simple Random Sample} - Each sample has same chance of being chosen
  \end{itemize}
\end{enumerate}
\textbf{Sources of Bias in Sample Survey:}
\begin{itemize}
  \item \textbf{Sampling Bias} - Sample not random or undercoverage
  \item \textbf{Non-response Bias} - No response from subject 
  \item \textbf{Response Bias} - Subject does not answer truthfully
\end{itemize}
\textbf{Elements of Good Experimental Study:}
\begin{itemize}
  \item Control comparison group
  \item Randomization: Eliminate lurking variables
  \item Blinding the study: Placebo
\end{itemize}


\section{03. Probabililty}
\subsection{Axioms of Probability}
\begin{enumerate}
  \item $0 \leq P(A) \leq 1$
  \item $P(S) = 1$
  \item If A and B are mutually exclusive, then $P(A \cup B) = P(A) + P(B)$. 
  $P(A \cap B) = 0$
  \item $P(A \cup B \cup C) = P(A) + P(B) + P(C) - P(A \cap B) - P(A \cap C) - 
  P(B \cap C) + P(A \cap B \cap C)$
\end{enumerate}
\textbf{For any events A and B:}
\begin{itemize}
  \item $P(A^{c}) = 1 - P(A)$
  \item $P(A \cup B) = P(A) + P(B) - P(A \cap B)$
  \item $P(A) = P(A \cap B) + P(A \cap B^{c})$
  \item A and B are \textbf{independent} if $P(A \cap B) = P(A)P(B)$
\end{itemize}

\subsection{Conditional Probability}
$P(A|B) = \frac{P(A \cap B)}{P(B)}$

\subsection{Law of Total Probabililty}
$P(A) = P(A \cap B_{1}) + ... + P(A \cap B_{n})$

\subsection{Bayes' Theorem}
$P(B_{i} | A) = \frac{P(A|B_{i})P(B_{i})}{P(A|B_{1})P(B_{1})...P(A|B_{n})P(B_{n})}$

\subsection{Epidemiological Terms}
\begin{itemize}
  \item \textbf{Sensitivity} - Given person has disease, prob. of positive test
  \item \textbf{Specificty} - Given person has no disease, prob. of negative test 
  \item \textbf{Prevalence} - No. of people with disease / Total population
\end{itemize}

\section{04. Random Variables}

\section{05. Sampling Distribution}

\section{06. Confidence Intervals}

\section{07. Hypothesis Testing}

\section{08. Linear Regression}

\section{09. R Code}

\begin{verbatim}
  test(2)
\end{verbatim}

\end{multicols}
\end{document}