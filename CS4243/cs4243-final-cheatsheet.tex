\documentclass{article}
\usepackage[a4paper, margin=3mm, landscape]{geometry}
\usepackage{multicol}
\usepackage{xcolor}
\usepackage{enumitem}
\usepackage{amsmath}
\usepackage{amsfonts}
\usepackage{listings}
\usepackage{soul}
\usepackage{graphicx}

\pdfinfo{
    /Title (CS4243.pdf)
    /Creator (TeX)
    /Producer (pdfTeX 1.40.0)
    /Author (Jason Qiu)
    /Subject (CS4243)
    /Keywords (CS4243, nus, cheatsheet, pdf)
}

\graphicspath{ {./img/} }

\pagestyle{empty}
\setcounter{secnumdepth}{0}
\setlength{\columnseprule}{0.25pt}

% Redefine section commands to use less space
\makeatletter
\renewcommand{\section}{\@startsection{section}{1}{0mm}%
    {-1ex plus -.5ex minus -.2ex}%
    {0.5ex plus .2ex}%x
{\normalfont\large\bfseries}}
\renewcommand{\subsection}{\@startsection{subsection}{2}{0mm}%
    {-1explus -.5ex minus -.2ex}%
    {0.5ex plus .2ex}%
{\normalfont\normalsize\bfseries}}
\renewcommand{\subsubsection}{\@startsection{subsubsection}{3}{0mm}%
    {-1ex plus -.5ex minus -.2ex}%
    {1ex plus .2ex}%
{\normalfont\small\bfseries}}%
\makeatother

% Adjust spacing for all itemize/enumerate
\setlength{\leftmargini}{0.5cm}
\setlength{\leftmarginii}{0.5cm}
\setlist[itemize,1]{leftmargin=2mm,labelindent=1mm,labelsep=1mm}
\setlist[itemize,2]{leftmargin=2mm,labelindent=1mm,labelsep=1mm,label=$\bullet$}

% Font
\renewcommand{\familydefault}{\sfdefault}

% Define colors for math formulas
\definecolor{myblue}{cmyk}{1,.72,0,.38}
\everymath\expandafter{\the\everymath \color{myblue}}

% Custom command for keywords
\definecolor{highlight}{RGB}{251,243,218}
\newcommand{\keyword}[2]{\sethlcolor{highlight}\hl{\textbf{#1}} - #2}
\newcommand{\ilkeyword}[1]{\sethlcolor{highlight}\hl{\textbf{#1}}}

% Define colors and style for code
\definecolor{codegreen}{rgb}{0,0.6,0}
\definecolor{codegray}{rgb}{0.5,0.5,0.5}
\definecolor{codered}{HTML}{CC241D}
\definecolor{backcolor}{rgb}{0.95,0.95,0.95}
\lstdefinestyle{codestyle}{
    backgroundcolor = \color{backcolor},
    commentstyle = \color{codegray},
    keywordstyle = \color{codered},
    stringstyle = \color{codegreen},
    basicstyle = \ttfamily,
    breakatwhitespace = false,
    showstringspaces = false,
    breaklines = true,
    showtabs = false,
    tabsize = 2
}
\lstset{style = codestyle}

% -----------------------------------------------------------------------
\begin{document}
\begin{multicols*}{4}
\footnotesize

% Title box
\begin{center}
    \fbox{
        \parbox{0.8\linewidth}{
            \centering \textcolor{black}{
                {\Large\textbf{CS4243}} \\
                \normalsize{AY23/24 Sem 2}} \\
                {\footnotesize \textcolor{gray}{github.com/jasonqiu212}}
        }
    }
\end{center}

\section{05. Segmentation}

\begin{itemize}
    \item Goal: Separate image into coherent regions
    \item Idea: \keyword{Clustering}{Group similar data points together}
    \item Challenges: What makes 2 points same/different? Choice of features (e.g. Color, Intensity, Position), Which clustering algorithm?
    \item \keyword{k-Means Clustering}{Iteratively re-assign points to nearest cluster center}
    \begin{enumerate}
        \item Given $K$, randomly initialize the cluster centers $c_1, \ldots, c_K$
        \item For each point $p_i$, find the closest $c_j$ and put $p_i$ into cluster $j$
        \item Given points in each cluster, set $c_j$ to be mean of points in cluster $j$
        \item Repeat, if $c_j$ have changed up to some threshold
    \end{enumerate}
    \begin{itemize}
        \item Pros: Simple, Converges to local min.
        \item Cons: Setting $K$, Sensitive to initial centers (Since k-means converges to local min.), Sensitive to outliers (Can add more clusters), Assumes spherical clusters (Fix with mean-shift)
    \end{itemize}
    \item \ilkeyword{Simple Linear Iterative Clustering (SLIC) Superpixels}
    \begin{itemize}
        \item \keyword{Superpixel}{Group of pixels that share common traits}
        \begin{itemize}
            \item Application: Inputs to other CV algo. since more compact representation with perceptual meaning
        \end{itemize}
        \item Num. of pixels: $n_{tp}$; Target num. of superpixels: $n_{sp}$
        \item Initial width of each superpixel: $s = \sqrt{\frac{n_{tp}}{n_{sp}}}$
        \item Features: $z = [r,g,b,x,y]$
        \item Color distance: $d_c = || \langle r_j,g_j,b_j \rangle - \langle r_i,g_i,b_i \rangle ||$
        \item Spatial distance: $d_s = || \langle x_j,y_j \rangle - \langle x_i,y_i \rangle ||$
        \item Scaling factors: $d_{cm}$ and $d_{sm}=s$ set as max. expected values of $d_c$ and $d_s$ respectively
        \item $D = \sqrt{(\frac{d_c}{d_{cm}})^2+(\frac{d_s}{d_{sm}})^2} = \sqrt{d_c^2 + (\frac{d_s}{s})^2 c^2}$
    \end{itemize}
    \begin{enumerate}
        \item Split img. into grid of size $s \times s$. Set cluster centers as lowest gradient position in $3 \times 3$ neighborhood from superpixel center to speed up convergence since initialize on value common to surrounding.
        \item For each cluster center, check distance to all pixels within $2s \times 2s$ neighborhood. Assign pixels to closest checked center.
        \item Update cluster centers using mean and repeat if not converged (Same as k-Means)
        \item Optional: Replace superpixel region by average value to create stained glass effect
    \end{enumerate}
    \begin{itemize}
        \item Modification of k-Means: Not random initialization, Compute pixel's distance only to closest set of cluster centers
        \item Can enforce connectivity and use other features too
    \end{itemize}
    \item \keyword{Mean-Shift Clustering}{Find local density maxima in feature space}
    \begin{itemize}
        \item \keyword{Attraction basin}{Region in feature space for which all trajectories of centroids lead to same mode}
        \item \keyword{Cluster}{All data points in attraction basin of a mode}
    \end{itemize}
    \begin{enumerate}
        \item For each data point:
        \begin{enumerate}
            \item Define window around and get centroid
            \item Shift window to centroid
            \item Repeat until window centroid stops moving
        \end{enumerate}
    \end{enumerate}
    \begin{itemize}
        \item Segmentation with Mean Shift: Do mean shift and merge pixels in same attraction basin
        \item Choosing window size: Trial and error, Sample points and use avg. dist. to knn. (Num. of neighbors needs to be large enough to ensure increase in density)
        \begin{itemize}
            \item Larger window size $\rightarrow$ Fewer clusters
        \end{itemize}
        \item Pros: No assumptions on cluster shape, 1 parameter, Finds variable num. of modes (vs. specified $k$ in k-Means), Robust to outliers
        \item Cons: Choosing $h$, Slow, Scales poorly with feature space dimension
        \item Optimizations:
        \begin{itemize}
            \item After each run of mean shift, assign all points within radius $r$ of end point to same cluster
            \item Assign all points within radius $c < r$ of search path to mode $\rightarrow$ More aggressive, less confident
        \end{itemize}
    \end{itemize}
\end{itemize}

\section{06. Texture}
\section{07. Keypoints}
\section{08. Descriptors}
\section{09. Homography}
\section{10. Optical Flow}
\section{11. Tracking}
\section{12. Deep Learning}

\end{multicols*}
\end{document}
